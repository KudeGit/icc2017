\documentclass[conference]{IEEEtran}
%\documentclass{sig-alternate}
% Some Computer Society conferences also require the compsoc mode option,
% but others use the standard conference format.
%
% If IEEEtran.cls has not been installed into the LaTeX system files,
% manually specify the path to it like:
% \documentclass[conference]{../sty/IEEEtran}

% Some very useful LaTeX packages include:
% (uncomment the ones you want to load)

% *** MISC UTILITY PACKAGES ***
%
%\usepackage{ifpdf}
% Heiko Oberdiek's ifpdf.sty is very useful if you need conditional
% compilation based on whether the output is pdf or dvi.
% usage:
% \ifpdf
%   % pdf code
% \else
%   % dvi code
% \fi
% The latest version of ifpdf.sty can be obtained from:
% http://www.ctan.org/pkg/ifpdf
% Also, note that IEEEtran.cls V1.7 and later provides a builtin
% \ifCLASSINFOpdf conditional that works the same way.
% When switching from latex to pdflatex and vice-versa, the compiler may
% have to be run twice to clear warning/error messages.

% *** CITATION PACKAGES ***
%
%\usepackage{cite}
% cite.sty was written by Donald Arseneau
% V1.6 and later of IEEEtran pre-defines the format of the cite.sty package
% \cite{} output to follow that of the IEEE. Loading the cite package will
% result in citation numbers being automatically sorted and properly
% "compressed/ranged". e.g., [1], [9], [2], [7], [5], [6] without using
% cite.sty will become [1], [2], [5]--[7], [9] using cite.sty. cite.sty's
% \cite will automatically add leading space, if needed. Use cite.sty's
% noadjust option (cite.sty V3.8 and later) if you want to turn this off
% such as if a citation ever needs to be enclosed in parenthesis.
% cite.sty is already installed on most LaTeX systems. Be sure and use
% version 5.0 (2009-03-20) and later if using hyperref.sty.
% The latest version can be obtained at:
% http://www.ctan.org/pkg/cite
% The documentation is contained in the cite.sty file itself.

% *** GRAPHICS RELATED PACKAGES ***
%
%\ifCLASSINFOpdf
  % \usepackage[pdftex]{graphicx}
  % declare the path(s) where your graphic files are
  % \graphicspath{{../pdf/}{../jpeg/}}
  % and their extensions so you won't have to specify these with
  % every instance of \includegraphics
  % \DeclareGraphicsExtensions{.pdf,.jpeg,.png}
%\else
  % or other class option (dvipsone, dvipdf, if not using dvips). graphicx
  % will default to the driver specified in the system graphics.cfg if no
  % driver is specified.
  % \usepackage[dvips]{graphicx}
  % declare the path(s) where your graphic files are
  % \graphicspath{{../eps/}}
  % and their extensions so you won't have to specify these with
  % every instance of \includegraphics
  % \DeclareGraphicsExtensions{.eps}
%\fi
% graphicx was written by David Carlisle and Sebastian Rahtz. It is
% required if you want graphics, photos, etc. graphicx.sty is already
% installed on most LaTeX systems. The latest version and documentation
% can be obtained at: 
% http://www.ctan.org/pkg/graphicx
% Another good source of documentation is "Using Imported Graphics in
% LaTeX2e" by Keith Reckdahl which can be found at:
% http://www.ctan.org/pkg/epslatex
%
% latex, and pdflatex in dvi mode, support graphics in encapsulated
% postscript (.eps) format. pdflatex in pdf mode supports graphics
% in .pdf, .jpeg, .png and .mps (metapost) formats. Users should ensure
% that all non-photo figures use a vector format (.eps, .pdf, .mps) and
% not a bitmapped formats (.jpeg, .png). The IEEE frowns on bitmapped formats
% which can result in "jaggedy"/blurry rendering of lines and letters as
% well as large increases in file sizes.
%
% You can find documentation about the pdfTeX application at:
% http://www.tug.org/applications/pdftex

% *** MATH PACKAGES ***
%
%\usepackage{amsmath}
% A popular package from the American Mathematical Society that provides
% many useful and powerful commands for dealing with mathematics.
%
% Note that the amsmath package sets \interdisplaylinepenalty to 10000
% thus preventing page breaks from occurring within multiline equations. Use:
%\interdisplaylinepenalty=2500
% after loading amsmath to restore such page breaks as IEEEtran.cls normally
% does. amsmath.sty is already installed on most LaTeX systems. The latest
% version and documentation can be obtained at:
% http://www.ctan.org/pkg/amsmath

% *** SPECIALIZED LIST PACKAGES ***
%
%\usepackage{algorithmic}
% algorithmic.sty was written by Peter Williams and Rogerio Brito.
% This package provides an algorithmic environment fo describing algorithms.
% You can use the algorithmic environment in-text or within a figure
% environment to provide for a floating algorithm. Do NOT use the algorithm
% floating environment provided by algorithm.sty (by the same authors) or
% algorithm2e.sty (by Christophe Fiorio) as the IEEE does not use dedicated
% algorithm float types and packages that provide these will not provide
% correct IEEE style captions. The latest version and documentation of
% algorithmic.sty can be obtained at:
% http://www.ctan.org/pkg/algorithms
% Also of interest may be the (relatively newer and more customizable)
% algorithmicx.sty package by Szasz Janos:
% http://www.ctan.org/pkg/algorithmicx

% *** ALIGNMENT PACKAGES ***
%
%\usepackage{array}
% Frank Mittelbach's and David Carlisle's array.sty patches and improves
% the standard LaTeX2e array and tabular environments to provide better
% appearance and additional user controls. As the default LaTeX2e table
% generation code is lacking to the point of almost being broken with
% respect to the quality of the end results, all users are strongly
% advised to use an enhanced (at the very least that provided by array.sty)
% set of table tools. array.sty is already installed on most systems. The
% latest version and documentation can be obtained at:
% http://www.ctan.org/pkg/array


% *** SUBFIGURE PACKAGES ***
%\ifCLASSOPTIONcompsoc
%  \usepackage[caption=false,font=normalsize,labelfont=sf,textfont=sf]{subfig}
%\else
%  \usepackage[caption=false,font=footnotesize]{subfig}
%\fi
% subfig.sty, written by Steven Douglas Cochran, is the modern replacement
% for subfigure.sty, the latter of which is no longer maintained and is
% incompatible with some LaTeX packages including fixltx2e. However,
% subfig.sty requires and automatically loads Axel Sommerfeldt's caption.sty
% which will override IEEEtran.cls' handling of captions and this will result
% in non-IEEE style figure/table captions. To prevent this problem, be sure
% and invoke subfig.sty's "caption=false" package option (available since
% subfig.sty version 1.3, 2005/06/28) as this is will preserve IEEEtran.cls
% handling of captions.
% Note that the Computer Society format requires a larger sans serif font
% than the serif footnote size font used in traditional IEEE formatting
% and thus the need to invoke different subfig.sty package options depending
% on whether compsoc mode has been enabled.
%
% The latest version and documentation of subfig.sty can be obtained at:
% http://www.ctan.org/pkg/subfig

% *** FLOAT PACKAGES ***
%
%\usepackage{fixltx2e}
% fixltx2e, the successor to the earlier fix2col.sty, was written by
% Frank Mittelbach and David Carlisle. This package corrects a few problems
% in the LaTeX2e kernel, the most notable of which is that in current
% LaTeX2e releases, the ordering of single and double column floats is not
% guaranteed to be preserved. Thus, an unpatched LaTeX2e can allow a
% single column figure to be placed prior to an earlier double column
% figure.
% Be aware that LaTeX2e kernels dated 2015 and later have fixltx2e.sty's
% corrections already built into the system in which case a warning will
% be issued if an attempt is made to load fixltx2e.sty as it is no longer
% needed.
% The latest version and documentation can be found at:
% http://www.ctan.org/pkg/fixltx2e

%\usepackage{stfloats}
% stfloats.sty was written by Sigitas Tolusis. This package gives LaTeX2e
% the ability to do double column floats at the bottom of the page as well
% as the top. (e.g., "\begin{figure*}[!b]" is not normally possible in
% LaTeX2e). It also provides a command:
%\fnbelowfloat
% to enable the placement of footnotes below bottom floats (the standard
% LaTeX2e kernel puts them above bottom floats). This is an invasive package
% which rewrites many portions of the LaTeX2e float routines. It may not work
% with other packages that modify the LaTeX2e float routines. The latest
% version and documentation can be obtained at:
% http://www.ctan.org/pkg/stfloats
% Do not use the stfloats baselinefloat ability as the IEEE does not allow
% \baselineskip to stretch. Authors submitting work to the IEEE should note
% that the IEEE rarely uses double column equations and that authors should try
% to avoid such use. Do not be tempted to use the cuted.sty or midfloat.sty
% packages (also by Sigitas Tolusis) as the IEEE does not format its papers in
% such ways.
% Do not attempt to use stfloats with fixltx2e as they are incompatible.
% Instead, use Morten Hogholm'a dblfloatfix which combines the features
% of both fixltx2e and stfloats:
%
% \usepackage{dblfloatfix}
% The latest version can be found at:
% http://www.ctan.org/pkg/dblfloatfix

% *** PDF, URL AND HYPERLINK PACKAGES ***
%
%\usepackage{url}
% url.sty was written by Donald Arseneau. It provides better support for
% handling and breaking URLs. url.sty is already installed on most LaTeX
% systems. The latest version and documentation can be obtained at:
% http://www.ctan.org/pkg/url
% Basically, \url{my_url_here}.

% *** Do not adjust lengths that control margins, column widths, etc. ***
% *** Do not use packages that alter fonts (such as pslatex).         ***
% There should be no need to do such things with IEEEtran.cls V1.6 and later.
% (Unless specifically asked to do so by the journal or conference you plan
% to submit to, of course. )

% correct bad hyphenation here
\hyphenation{op-tical net-works semi-conduc-tor}

\begin{document}

\title{Anomaly Detection in Wireless Networks \\using Controller Logs}

% author names and affiliations

\author{
    \IEEEauthorblockN{Taha Hajar\IEEEauthorrefmark{1}, 
George Christoudoulou\IEEEauthorrefmark{2},
Davide Cuda\IEEEauthorrefmark{2},
Domenico Ficara\IEEEauthorrefmark{2},
Lorenzo Granai\IEEEauthorrefmark{2}
}\\
\IEEEauthorblockA{\IEEEauthorrefmark{1} EPFL, Lausanne, Switzerland, taha.hajar@epfl.ch}
\IEEEauthorblockA{\IEEEauthorrefmark{2} Cisco System, Switzerland, \{georgchr, dcuda, dficara, lgranai\}@cisco.com}
}

\maketitle

% As a general rule, do not put math, special symbols or citations
% in the abstract
\begin{abstract}
With the growing demand for wireless connectivity, enterprises are investing more in supporting wireless access in their large scale networks. However, the deployment of wireless access introduces a new level of complexity and opens the door for new problems in the network. While manual troubleshooting is time consuming and requires much experience, an indispensable need for smart automated troubleshooting is emerging . In this paper, a machine learning solution is proposed to automate anomaly detection in wireless networks. The solution uses the network logs to derive sequences of network events in real-time. Two sequence learning techniques are then applied and combined to detect abnormal sequences of network events. These methods use a novel combination of features and distance metric compatible with the unstructured nature of the sequences. Furthermore, F-scores and ROC curves are used to compare the performances of different models. Finally, this paper presents real case scenarios where this innovative approach improves network management by automatically gaining real-time useful information on wireless networks status.

\end{abstract}

% no keywords


\section{Introduction}
% no \IEEEPARstart
With the explosion of the number of smartphone and wireless device connected to the Internet, WiFi network becomes one of the most critical network segment. According to reports published by Dimension Data, network access ports were 80\% wired and 20\% wireless in 2013. With the wireless access overtaking the wired access, the same source also predicted that the access ports will eventually become 20\%wired and 80\% wireless in a couple of years [1]. Nowadays, large enterprises such as universities and big companies, can have up to thousands of access points (AP) managed under the same network. Consequently, maintaining and assuring a good health of the network become crucial, critical and challenging. In this context, timely identifying issues in the wireless context is of primary importance to ensure the required quality of experience for users. However, the large number of 802.11 protocols, their complex state machine and the variance of the wireless medium make this task even harder than usual. 

Numerous networking issues are reported to networks� administrators daily. Although there exist tens of useful troubleshooting tools, diagnosing network problems has always been a challenging task. Furthermore, as the network size, its complexity, and the number of features it supports increases, the number of dimensions a network engineer has to consider while solving a problem drastically increases. This puts more pressure on the task of monitoring and solving network issues. Few examples of the features needed to be considered are physical connectivity, network configuration, router configurations and status (queue status, CPU load, memory usage, etc�), network protocols� state machines and the interaction between the application layer and the network layer. Therefore, network troubleshooting usually requires skilled network engineers that leverage years of experience.

In a typical scenario When network users experience a network problem, they usually contact the network administrator who starts investigating the network logs in order to understand what happened. However, the large number of logs and the concurrency feature of wireless networks make this task quite complex.

In this context, machine learning techniques can help in automating the log analysis task. Nowadays, the trend is to make the machine learn by itself how to assess the situation in the system by using available data. This technique is very efficient, especially when big data is available since the machine processes the data much faster than a human. The network logs, being a rich source of data, is then a potential source of useful data that can be used in automating network troubleshooting.

Therefore, in this paper we study the possibility of building a machine learning solution that can identify network anomalies automatically without human intervention. In the proposed solution, the machine processes the network logs in real-time and monitors the behavior of the network. Whenever an anomaly is detected, it informs the administrator and points her to the logs reflecting the anomaly. With this solution, the administrator can proactively take some action to limit the window of time in which users perceive some connectivity problems instead of inspecting the problem.


The paper is organized as follows. In Section \ref{sec:related}, we provide a concise summary of
the most pertinent related work and we highlight how this approach is different from the proposed
solutions. In Section \ref{sec:problem} we formalize the problem and... 

\section{Related works}
Automated troubleshooting for wired networks is not a new topic. Many studies tackled this issue using myriad analytic and learning techniques. However, the introduction of wireless was a game changer. Traditional tools designed for wired networks are not able to cover the problems of wireless features. Thus, recent studies started attacking the automation of wireless troubleshooting from different angles. Nevertheless, to the best of our knowledge, none of these studies addresses the possibility of using machine learning on network logs. The closest works are machine learning tools that use measurement data to predict the source causes of a bad performance or detect anomalies in the wireless network.

A research in [8] studies the effect of media access dynamics, mobility management and 802.11 protocols on the network performance from the perspective of the physical layer to the transport layer. The authors built an automated tool that collects measurements of delays using a pre-installed infrastructure of APs for testing. The tool uses these measurements along with several trained models to infer hidden delays and detect their sources.

Two famous solutions for wireless enterprise network monitoring and troubleshooting are WifiProfiler [9] and DAIR [10]. Both solutions detect wireless problems and predict the network performance using data collected from the client�s perspective without assuming any special capabilities in the infrastructure. However, the main disadvantage of these solutions is that they are instrumented to test for specific anomalies. They do not use any artificial intelligence or machine learning technique.

While log analysis is not widely used in network troubleshooting, it has been immensely used to understand the behaviors of other computer systems. In some studies such as [13,14], the common feature extraction techniques, used in the natural language processing field, were applied to classify log files.

In other log analysis studies, log files are used to study certain event correlation in order to discover new patterns and causalities between events.For example, two studies [15,16] follow a data-driven framework to acquire all needed data from a large set of log files. This approach first assimilates the semantics of uniform events and categorizes the events into common base event (CBE) format. After recognizing similar messages as the same CBE, interesting patterns are discovered by exploiting temporal dependencies among events.

In this project, we combine both log analysis approaches, the pattern discovery and the feature extraction methods. The discovered patterns are used as features to learn the different behaviors in the network and detect anomalies.

\section{Problem overview}
In order to identify anomaly, we focus on a specific network configuration. For this network topology we collect logs, that we analyze in order to idetify anomalies.

\subsection{Network architecture}
We consider a typical enterprise or campus wireless access network. Usually a single Cisco Wireless Controller as the Cisco 8540 Wireless Controller or the Cisco Catalyst 3850 can support thousand of Access Points (AP) and ten of thousands of clients~\cite{WLC-datasheets-site}. To simplify configurations and management, 
We consider lightweight APs, i.e., APs cannot act independently of a wireless controller (RFC 5412). In order to monitor and identify issues with this network while minimizing the amount of data processed, we focus on the data collected by the WLC. Collecting data on the WLC also void synchronization issues as well as .... 

We focus mainly on parsing and analyzing the network logs. This is because global show commands usually show a specific summarized information, in precise time instant. On the other hand, the log files cover the whole sequence of events occurring in the network. Due to their complexity and lengths, the log files are hard to be processed by a human being. However, our hypothesis is that machine learning can help us find the interesting information efficiently from these log files. In our case, we are interested in detecting abnormal sequences of network events.
\subsection{Problem statement}
In order to formulate the problem, we introduce some notations and definitions. First, we consider that every log entry represents a network event. A log entry \textbf{\textit{x}} consists of a timestamp \(t_x\), a list of mac addresses \(d_x\) of the devices involved in the event and a message \(m_x\) describing the action that took place. In the preprocessing phase, we assign a unique ID \textbf{\textit{ID}} for every possible action. Then, we map every message to the corresponding \textbf{\textit{ID}}. An event \(e_x\) is thus defined as the tuple containing the timestamp, the mac address list and the event ID corresponding to the log entry x:
\[e_x = (t_x,d_x,ID_x)\]
We also denote the ordered sequence s of n consecutive events by:
\[s = <e_1,e_2, ... e_n>\]
In this project, the sequences to be labeled are derived based on the mac address involved in the event. In other words, for every mac address that appears in the logs, we construct a sequence of events in which it is involved. Thus, for a dataset where m mac addresses are seen, m sequences are constructed and processed. We denote the sequence corresponding to the mac address \textbf{\textit{i}} by: 
\[s_i = <\forall e_x | i \in d_x> \]

Depending on the activity of the corresponding device, the sequence might have gaps in which consecutive events are not contiguous. Therefore, each of the sequences is split into contiguous subsequences. We denote a contiguous subsequence by \(s'_{i,j}\) where i is the corresponding mac address and j is the order of the subsequence. 

Based on the previous definitions, We formally define the problem as follows. Given the set of all contiguous subsequences derived from the logs, we need to identify the abnormal subsequences. 





\section{Model and Methodology}
\subsection{Pattern discovery}
\subsection{Sequence learning}
\subsection{Learning algorithm}



\section{Testing and Results}
\section{Conclusions}
Stress main idea. Usage and impact. Alternatives and future work.\\
Usage: Network monitor, debugging tool, assess log system\\
Future work: noise resilient, anomaly classification


% An example of a floating figure using the graphicx package.
% Note that \label must occur AFTER (or within) \caption.
% For figures, \caption should occur after the \includegraphics.
% Note that IEEEtran v1.7 and later has special internal code that
% is designed to preserve the operation of \label within \caption
% even when the captionsoff option is in effect. However, because
% of issues like this, it may be the safest practice to put all your
% \label just after \caption rather than within \caption{}.
%
% Reminder: the "draftcls" or "draftclsnofoot", not "draft", class
% option should be used if it is desired that the figures are to be
% displayed while in draft mode.
%
%\begin{figure}[!t]
%\centering
%\includegraphics[width=2.5in]{myfigure}
% where an .eps filename suffix will be assumed under latex, 
% and a .pdf suffix will be assumed for pdflatex; or what has been declared
% via \DeclareGraphicsExtensions.
%\caption{Simulation results for the network.}
%\label{fig_sim}
%\end{figure}

% Note that the IEEE typically puts floats only at the top, even when this
% results in a large percentage of a column being occupied by floats.


% An example of a double column floating figure using two subfigures.
% (The subfig.sty package must be loaded for this to work.)
% The subfigure \label commands are set within each subfloat command,
% and the \label for the overall figure must come after \caption.
% \hfil is used as a separator to get equal spacing.
% Watch out that the combined width of all the subfigures on a 
% line do not exceed the text width or a line break will occur.
%
%\begin{figure*}[!t]
%\centering
%\subfloat[Case I]{\includegraphics[width=2.5in]{box}%
%\label{fig_first_case}}
%\hfil
%\subfloat[Case II]{\includegraphics[width=2.5in]{box}%
%\label{fig_second_case}}
%\caption{Simulation results for the network.}
%\label{fig_sim}
%\end{figure*}
%
% Note that often IEEE papers with subfigures do not employ subfigure
% captions (using the optional argument to \subfloat[]), but instead will
% reference/describe all of them (a), (b), etc., within the main caption.
% Be aware that for subfig.sty to generate the (a), (b), etc., subfigure
% labels, the optional argument to \subfloat must be present. If a
% subcaption is not desired, just leave its contents blank,
% e.g., \subfloat[].


% An example of a floating table. Note that, for IEEE style tables, the
% \caption command should come BEFORE the table and, given that table
% captions serve much like titles, are usually capitalized except for words
% such as a, an, and, as, at, but, by, for, in, nor, of, on, or, the, to
% and up, which are usually not capitalized unless they are the first or
% last word of the caption. Table text will default to \footnotesize as
% the IEEE normally uses this smaller font for tables.
% The \label must come after \caption as always.
%
%\begin{table}[!t]
%% increase table row spacing, adjust to taste
%\renewcommand{\arraystretch}{1.3}
% if using array.sty, it might be a good idea to tweak the value of
% \extrarowheight as needed to properly center the text within the cells
%\caption{An Example of a Table}
%\label{table_example}
%\centering
%% Some packages, such as MDW tools, offer better commands for making tables
%% than the plain LaTeX2e tabular which is used here.
%\begin{tabular}{|c||c|}
%\hline
%One & Two\\
%\hline
%Three & Four\\
%\hline
%\end{tabular}
%\end{table}


% Note that the IEEE does not put floats in the very first column
% - or typically anywhere on the first page for that matter. Also,
% in-text middle ("here") positioning is typically not used, but it
% is allowed and encouraged for Computer Society conferences (but
% not Computer Society journals). Most IEEE journals/conferences use
% top floats exclusively. 
% Note that, LaTeX2e, unlike IEEE journals/conferences, places
% footnotes above bottom floats. This can be corrected via the
% \fnbelowfloat command of the stfloats package.


% use section* for acknowledgment


% trigger a \newpage just before the given reference
% number - used to balance the columns on the last page
% adjust value as needed - may need to be readjusted if
% the document is modified later
%\IEEEtriggeratref{8}
% The "triggered" command can be changed if desired:
%\IEEEtriggercmd{\enlargethispage{-5in}}

% references section

% can use a bibliography generated by BibTeX as a .bbl file
% BibTeX documentation can be easily obtained at:
% http://mirror.ctan.org/biblio/bibtex/contrib/doc/
% The IEEEtran BibTeX style support page is at:
% http://www.michaelshell.org/tex/ieeetran/bibtex/
%\bibliographystyle{IEEEtran}
% argument is your BibTeX string definitions and bibliography database(s)
%\bibliography{IEEEabrv,../bib/paper}
%
% <OR> manually copy in the resultant .bbl file
% set second argument of \begin to the number of references
% (used to reserve space for the reference number labels box)
% \begin{thebibliography}{1}

% \bibitem{IEEEhowto:kopka}
% H.~Kopka and P.~W. Daly, \emph{A Guide to \LaTeX}, 3rd~ed.\hskip 1em plus
%   0.5em minus 0.4em\relax Harlow, England: Addison-Wesley, 1999.

% \end{thebibliography}

\end{document}
